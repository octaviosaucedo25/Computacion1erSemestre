\documentclass[10pt,a4paper]{book}

\usepackage[utf8]{inputenc}

\title{Mi primer documento escrito en LaTeX}
\author{Octavio Saucedo Avila}

\begin{document}
\maketitle
\tableofcontents

\part{Lógica proposicional}
\chapter{Proposiciones}
\section{Conceptos primitivos}
\textbf{Definición}. Una proposición es cualquier afirmación de la cual podemos decir si es falsa o verdadera. 
\section{Conectivos lógicos}
Conjunción, disyunción, implicación, doble implicación y disyunción exclusiva. 
\subsection{La conjunción}
Aprovecho este espacio para aclarar que es mi primera vez escribiendo en LaTeX y quise explorar algunas otras opciones además de la plantilla que nos proporcionaron. 
\subsection{La identidad de Euler}
$$ e^{i \pi} + 1 = 0 $$ 
Esta igualdad matemática es considerada una de las más hermosas porque involucra a las constantes más importantes de esta ciencia: el número de Euler, el número $\pi$, la unidad imaginaria $i$, el elemento neutro de la suma y el elemento neutro del producto. 

Se trata de la forma de Euler para escribir un número complejo cuando $\theta = \pi$. Es decir, al partir de la siguiente relación: $$ e^{i \theta} = \cos{\theta} + i\sin{\theta} $$
sustituimos $\theta$ por $\pi$ y tendremos lo siguiente:
$$ e^{i \pi} = \cos{\pi}+i\sin{\pi}$$
$$ e^{i \pi}=1+i(0)\Rightarrow e^{i \pi}=1\Rightarrow  e^{i \pi} + 1 = 0 $$
obteniendo así de manera informal la famosa identidad de Euler. 
\subsection{La ecuación de onda}
Una relación que aún no comprendo, pero se ve muy elegante; y de la cual tuve que investigar su forma de escribirse en LaTeX es la \textbf{ecuación de onda}:
$$ \nabla^{2} \psi(\vec{r}, t)=\frac{1}{v^{2}} \frac{\partial^{2} \psi(\vec{r}, t)}{\partial t^{2}} $$


\end{document}